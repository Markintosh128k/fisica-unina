\chapter{Numeri Complessi}\label{numeri-complessi}

\defn{Definizione}{
In \(\mathbb{R}^2\) definiamo due operazioni: \[
\begin{aligned}
+ &:(x_1 , y_1)+(x_2 , y_2)=(x_1 + x_2 , y_1 + y_2) \in \mathbb{R}^2 \\
\cdot &: (x_1 , y_1)\cdot (x_2 , y_2)=(x_{1} x_{2} - y_{1} y_{2}, y_{1}x_{2} + x_{1}y_{2}) \in \mathbb{R}^2
\end{aligned}
\] Queste operazioni identificano un \textbf{campo}.
}

\section{\texorpdfstring{\(\mathbb{C}\) --- Campo dei numeri
complessi}{\textbackslash mathbb\{C\} --- Campo dei numeri complessi}}\label{mathbbc-campo-dei-numeri-complessi}

\defn{Definizione}{
\[\boxed{
\begin{array}{l}
(0,1)\cdot(0,1) = (-1,0) = -1\\
i : \text{ è l'unità immaginaria }= (0,1)
\end{array}
}\]

Per \(z \in \mathbb{C}, \; z=(a,b)\): \[
\begin{array}{l}
(a,b) = (a,0) + (0,1)(b,0)\\
\boxed{z=a+ib} \quad \text{Forma algebrica}
\end{array}
\]
}

\newpage

\section{Complesso Coniugato}\label{complesso-coniugato}

\defn{Definizione}{
Se \(z = x + iy\) si chiama \textbf{complesso}, il suo \textbf{coniugato} è \(\overline{z} = x - iy\).
}

\ex{
Esempio\\
\[
\frac{3 - 2i}{5 + i} \cdot \frac{\overline{z}}{\overline{z}} = \frac{(3-2i)(5-i)}{(5+i)(5-i)} =\frac{(3-2i)(5-i)}{26}
\]
}

\section{Modulo}\label{modulo}

\defn{Definizione}{
Modulo \[|z| = \sqrt{x^2 + y^2}\]
}

\ex{
Esempio\\
\[\begin{array}{l}
z_1 = 2i \quad z_2 = 3 - i \\
|z_1 - z_2| = |2i - 3 + i| = |-3 + 3i| = 3\sqrt{2}
\end{array}\]
}

\section{Forma trigonometrica}\label{forma-trigonometrica}

\defn{Definizione}{
Forma trigonometrica
\[
\begin{array}{l}
x = \rho \cos \theta , \quad y = \rho \sin \theta \\
\tan \theta = \frac{y}{x} \\
\text{ALLORA} \quad z = \rho[\cos\theta+i\sin\theta]
\end{array}
\]

\(\theta\) si chiama \textbf{argomento del numero complesso} e si indica
con\\
\[arg(z) = \{\theta + 2k\pi, \; k \in \mathbb{Z}\}\]

\(\theta\) non è univocamente determinato.\\
Si chiama \textbf{argomento principale} \(\theta \in ]-\pi, \pi]\), che
diventa univocamente determinato.\\
Abbiamo così un cambio in \textbf{coordinate polari}, \(\rho\) e
\(\theta\).
}

\begin{center}
\begin{tabular}{p{0.45\linewidth} p{0.45\linewidth}}
\hline
\textbf{Cartesiane} & \textbf{Polari} \\
\hline
\(z = 2+2\sqrt{3}i\) & \(x=\rho \cos \theta\) \\
\(\rho = 2 \sqrt{4}=4\) &
\(2=4\cos\theta \Rightarrow \cos \theta = \frac{1}{2}\) \\
\(z=4(\cos \frac{\pi}{3}+i \sin \frac{\pi}{3})\) &
\(y=\rho \sin \theta=2\sqrt{3}=4\sin \theta \Rightarrow \sin \theta = \frac{\sqrt{3}}{2}\) \\
\hline
\end{tabular}
\end{center}

\[Arg(z)=\frac{\pi}{3} \qquad arg(z)=\left[\frac{\pi}{3} + 2k\pi, \; k \in \mathbb{Z}\right]\]

\textbf{Perché la forma trigonometrica è fondamentale?}\\
Perché, ad esempio, per calcolare \((2 - 3i)^5\) dovremmo fare 5 volte
il prodotto,\\
mentre con la forma trigonometrica è molto più semplice.

\prop{
Proprietà\\
\[
\begin{array}{l}
z_1 = \rho_1 [\cos\theta_1 + i \sin\theta_1] \\
z_2 = \rho_2 [\cos\theta_2 + i \sin\theta_2] \\
z_1 \cdot z_2 = \rho_1 \rho_2 [\cos(\theta_1 + \theta_2) + i \sin(\theta_1 + \theta_2)]
\end{array}
\]
}

\section{Formula di de Moivre}\label{formula-di-de-moivre}

\defn{Definizione}{
\[
z^n = \rho^n [\cos(n\theta) + i \sin(n\theta)]
\]
}

\ex{
Esempio\\
\[
\begin{aligned}
&(-1+i)^5 \quad (-1+i)=z\\
&\rho = \sqrt{2}\\
&Arg(z) = \frac{3}{4}\pi\\
&z=\sqrt{2}[\cos{\frac{3}{4}\pi}+i\sin{\frac{3}{4}\pi}]\\
&z^5=\sqrt{2}[\cos{\frac{15}{4}\pi}+i\sin{\frac{15}{4}\pi}]= 4\sqrt{2}[\frac{\sqrt{2}}{2}-\frac{\sqrt{2}}{2}i]=\\
&= 4(1-i)
\end{aligned}
\]
}

\section{Forma esponenziale}\label{forma-esponenziale}

\defn{Formula di Eulero}{
\[e^{i\theta} = \cos \theta + i \sin \theta\]

\[e^{i\pi} + 1 = 0 \quad \text{(Identità di Eulero, molto famosa)}\]

Quindi:\\
\[z = \rho [\cos \theta + i \sin \theta] \Rightarrow \boxed{z = \rho e^{i\theta}}\]

Non conosciamo ancora il significato di questa funzione, lo vedremo in
\emph{Metodi Matematici per la Fisica (Analisi 3)}.
}

\section{Radice n-esima di un
complesso}\label{radice-n-esima-di-un-complesso}

\defn{Definizione}{
Sia \(n \in \mathbb{N}\) e \(z \in \mathbb{C}\). Si chiama
\textbf{radice n-esima di \(z\)} ogni numero complesso \(w\) tale che:
\[
\boxed{w^n = z}
\]

Scriviamo \[
z = \rho (\cos \theta + i \sin \theta),
\] e cerchiamo \(w\) tale che \[
w^n = r (\cos \phi + i \sin \phi)
\]

Applicando la \textbf{formula di de Moivre}, le radici si possono
scrivere come: \[
w_k = r(\cos \phi_k + i \sin \phi_k), \quad k = 0,1,\dots,n-1
\] dove \(\phi_k\) è definito dal sistema: \[
w^n=z \iff 
\left\{
\begin{array}{l}
r = \sqrt[n]{\rho} \\
\phi_k = \frac{\theta + 2 k \pi}{n}, \quad k = [0,1,2,\dots,n-1]
\end{array}
\right.
\]
}

\ex{
Esercizio\\
\(z^4 = 1 \Rightarrow z = \sqrt[4]{1}\)\\
Dobbiamo calcolare le \textbf{radici quarte di 1}:\\
\(1 = 1 \cdot e^{i0}\) con \(\rho = 1\) e \(\theta = 0\).
\[
\begin{array}{l}
|w_k| = \sqrt[4]{1} = 1 \\[4pt]
\phi_k = \frac{\theta + 2k\pi}{4} = \frac{k\pi}{2}, \quad k = 0,1,2,3
\end{array}
\]

\begin{center}
\begin{tabular}{c | c | l}
\(k\) & \(\phi_k\) (angolo) & \(w_k\) (radice) \\
\hline
0 & \(0\) & \(w_0 = \cos 0 + i\sin 0 = 1\) \\
1 & \(\frac{\pi}{2}\) & \(w_1 = \cos \frac{\pi}{2} + i\sin \frac{\pi}{2} = i\) \\
2 & \(\pi\) & \(w_2 = \cos \pi + i\sin \pi = -1\) \\
3 & \(\frac{3\pi}{2}\) & \(w_3 = \cos \frac{3\pi}{2} + i\sin \frac{3\pi}{2} = -i\)
\end{tabular}
\end{center}

Le radici, quando le troviamo, risultano tutte \textbf{sulla
circonferenza}.
}